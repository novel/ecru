\documentclass{article}
\usepackage{verbatim} 
\title{Introduction to Ecru}
\author{Roman Bogorodsky (bogorodskiy@gmail.com)}
\date{}
\begin{document}
   \maketitle
   \section{About}
    Ecru is a command-line LiveJournal client which pretends to be flexible
    and follow Unix way.
 
    You might consult README file in the Ecru distribution in order to get
    information about dependencies and build process.

    \section{Getting started}
      \subsection{Configuration}
       Ecru comes with a tool {\tt ecru-config} which is responsible for
       configuration related issues. To generate initial configuration schema,
       execute {\tt ecru-config } with {\tt -g } argument: { \tt ecru-config -g }. It
       will prompt for your livejournal username and md5 hash of the password.
       On Linux systems you can generate md5 hash of your password using
       {\tt md5sum} utility and on FreeBSD you might use {\tt md5 -s}. So, in order
       to avoid interactivity, you might do the following:
  
	\begin{verbatim}
		ecru-config -g -u username -p `md5 -s mypass`
	\end{verbatim}
	
	Consider that it might be insecure to leave your plain text password in the
	shell history.

	A new directory {\tt ecru.new} will be created by {\tt ecru-config}. Just 
	{\tt cp -r } it to {\tt ~/.ecru} and you're done with it.
       \subsection{Posting}
	A tool called {\tt ecru-post } is responsible for posting new entries. It can be used
	either in interactive or command line mode. Let's start with the interactive mode first.

	If you type {\tt ecru-post } without any argument it will load {\tt \$EDITOR }. As soon as you
	finish typing, save the file and quit from the editor, the post will be submitted and
	{\tt ecru-post } will return an URL of the new post.

	To add subject to the post, you should add a 'subject' keyword to the top of the text, like:
	\begin{verbatim}
		subject: hey, this is a subject

		the text of a post goes here
		one more line just to make it look more real
    	\end{verbatim}

	You might add other keywords like current mood and stuff, for example:
	
	\begin{verbatim}
		subject: another post
		current_mood: good
		current_music: The Beatles - Let It Be
	
		Some very kewlz post, YA RLY.
	\end{verbatim}

	Please note a blank line separating headers part and the body, it's required. It could be skipped
	only if your post has no headers at all.

	As it was mentioned before, it's possible to post in non-interactive mode. It could be done this way:
	
	\begin{verbatim}
	echo "body of the post" | ecru-post -f - \
		-s "subject of the post" \
		-Dcurrent_mood=good -Dcurrent_music="`music.sh`"
	\end{verbatim}

	It will post a new entry without invoking editor. As you might have noticed, {\tt -s } is used to set the subject
	and {\tt -D} is used to set properties. 

	Note that you can use {\tt -s} and {\tt -D} in an interactive mode as well, however command line arguments has
	lower priority than the ones defined in the text. E.g. if you executed {\tt ecru-post -s "some subject"} and
	didn't provide 'subject:' keyword in the text, the subject of you post will be "some subject". However, if
	you execute {\tt ecru-post -s "some subject"} and add "subject:" keyword to the text, like "subject: cooler
	subject", the subject of you post will be "cooler subject". The same is valid for {\tt -D} as well.

	\subsection{Templates}
	Before {\tt ecru-post} invokes the editor it loads the contents of the {\tt ~/.ecru/templates/default} file. 
	That's where 'subject:' line comes from with the default configuration. You might alter 
	{\tt ~/.ecru/templates/default} template for your needs. You might create new templates and place it into
	{\tt ~/.ecru/template/} directory and pass their name to {\tt -t} arg for {\tt ecru-post}. For example,
	if you created {\tt ~/.ecru/template/mytemplate} you call {\tt ecru-post -t mytemplate} to use it.

	\subsection{Post operations: list, edit, delete}
	A tool {\tt ecru-list} could be used to last of recent posts. In first column it shows item id (important!),
	then few first chars from the post body and post time. If you pass {\tt -s} argument, it will show post url
	as well.

	You can delete posts using {\tt ecru-delete} posts, e.g. if you want to delete posts with ids 10 and 11 you
	do {\tt ecru-delete 10 11}. As a reminder: you can look up an id in the first column of {\tt ecru-list} 
	output.

	To edit post with id 10 you need to execute {\tt ecru-edit 10}. 

	To obtain an info about a post with id 10 you need to execute {\tt ecru-info 10}. By the way, there's a special
	id "-1" which always refers to the latest post in your journal. For example: {\tt ecru-info -- -1} will
	show an info about the latest post. Note "--" - it's used to getopt didn't think "-1" is an argument.
     \section{Advanced topics}
       \subsection{Configuration profiles}
	Ecru supports having several config profiles (which could be useful if you have few accounts). If you want to
	add a new profile, just create a new configuration file in {\tt ~/.ecru/}. Its name should end with
	{\tt .conf}. 
	Now if you run {\tt ecru-config -l} you should see a list of configurations, in our example it should be
	'default.conf' marked with asterisk and the configuration file you just created. Asterisk (*) marks
	currently active configuration profile. To change current configuration file you should do (assuming
	you named file example.conf): 
	\begin{verbatim}
		ecru-config -s example.conf
	\end{verbatim}

	Now {\tt ecru-config -l} should show example.conf marked with asterisk.
\end{document}
