\documentclass[english]{article}
\usepackage[latin1]{inputenc}
\usepackage{babel}
\usepackage{verbatim}
\usepackage[fancy]{ecru}

\setVersion{0.1.0}

\begin{document}

\begin{Name}{1}{ecru-edit}{Roman Bogorodskiy}{ecru}{ecru-edit}
	\Prog{ecru-edit} is a tool for editing posts.
\end{Name}

\section{Synopsis}

\Prog{ecru-edit} \Arg{id}

\Prog{ecru-edit} \Opt{-v}


\section{Description}
\Prog{ecru-info} is a tool for editing posts. It takes id of an item to
edit as argument. It fetches a post and opens it in editor specified by
\texttt{EDITOR} environment variable or \texttt{vi} if it's not defined.
When you are ready, save the file and leave (quit) the editor, the entry
will be updated. You can get list of the item ids by calling \Cmd{ecru-list}{1}.

\section{Options}

\begin{Description}\setlength{\itemsep}{0cm}
\item[\Opt{-v}] Show ecru version.
\end{Description}

\section{Environment}
\begin{Description}\setlength{\itemsep}{0cm}
\item[EDITOR] Specifies the editor to use.
\end{Description}

\section{See Also}

\Cmd{ecru-post}{1}, \Cmd{ecru-list}{1}, \Cmd{ecru-delete}{1}, \Cmd{ecru-info}{1}
