\documentclass[english]{article}
\usepackage[latin1]{inputenc}
\usepackage{babel}
\usepackage{verbatim}
\usepackage[fancy]{ecru}

\setVersion{0.1.0}

\begin{document}

\begin{Name}{1}{ecru-post}{Roman Bogorodskiy}{ecru}{ecru-post}
	\Prog{ecru-post} is a tool for posting new entries to the journal.
\end{Name}

\section{Synopsis}

\Prog{ecru-post} \oOptArg{-D}{prop=value} \oOptArg{-s}{subject} \oOptArg{-u}{journal}
	\oOptArg{-t}{template} \oOptArg{-f}{filename}

\Prog{ecru-post} \Opt{-v}

\section{Description}
\Prog{ecru-post} is a tool for adding new entries to the journal. As you can see,
all options are optional. Unless \Opt{-f} is given, it starts editor specified
via EDITOR environment variable or uses \texttt{vi} if it's not defined.

Read 'ecru introduction' for description of the post structure.

If you decide not to submit post, you can delete the post body and quit the editor,
post will not be submitted.
\section{Options}

\begin{Description}\setlength{\itemsep}{0cm}
\item[\OptArg{-D}{prop=value}] Set property \texttt{prop} to value \texttt{value}.
This option can be used multiple times.
\item[\OptArg{-s}{subject}] Set the subject of the entry to \Arg{subject}.
\item[\OptArg{-u}{journal}] Set journal to post in.
\item[\OptArg{-t}{template}] Specify template file to pre-load.
\item[\OptArg{-f}{filename}] Instead of opening \texttt{\$EDITOR}, load contents of the post
from \Arg{filename}. If \Arg{filename} equals to \texttt{-}, read contents 
from \texttt{stdin}.
\end{Description}

\section{Environment}
\begin{Description}\setlength{\itemsep}{0cm}
\item[EDITOR] Specifies the editor to use.
\end{Description}

\section{See Also}

\Cmd{ecru-edit}{1}, \Cmd{ecru-list}{1}, \Cmd{ecru-delete}{1}, \Cmd{ecru-info}{1}
