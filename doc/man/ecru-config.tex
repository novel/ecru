\documentclass[english]{article}
\usepackage[latin1]{inputenc}
\usepackage{babel}
\usepackage{verbatim}
\usepackage[fancy]{ecru}

\setVersion{0.1.0}

\begin{document}

\begin{Name}{1}{ecru-config}{Roman Bogorodskiy}{ecru}{ecru-config}
	\Prog{ecru-config} is a tool for managing ecru configuration.
\end{Name}

\section{Synopsis}

\Prog{ecru-config} \Arg{config\_variable}

\Prog{ecru-config} \oOpt{-l} \oOptArg{-s}{profile}

\Prog{ecru-config} \Opt{-v}

\Prog{ecru-config} \OptArg{-h}{hook}

\section{Description}
\subsection{Common}
\Prog{ecru-config} is a tool for managing ecru configuration. Actually, since
all conifguration files are plain text, it's possible to modify them directly
without using \Prog{ecru-config}, however \Prog{ecru-config} can save  your
time.

By default \Prog{ecru-config} prints value of the config variable given by
\Arg{config\_variable} argument. \Arg{config\_variable} is an XPath-like 
expression of the configuration varible name.

For example, if the configuration file looks this way:

\begin{verbatim}
config: {
        account:
        {
                login = "somelogin";
                password = "somepass";
        };
};
\end{verbatim}

you can do the following:

\Prog{ecru-config} \Arg{config.account.login}

and it will print out "somelogin".

\subsection{Profiles}
Ecru supports having several configuration profiles. Configuration profile
is basically a plain text file with \texttt{.conf} extension which is located
inside \texttt{\$HOME/.ecru} directory. When executed with \Opt{-l} option,
\Prog{ecru-config} prints out a list of found configuration profiles marking
the current one with asterisk.

You can switch configuration profiles using \texttt{-s} option. 
It changes current configuration to the value given in \texttt{profile}. \texttt{profile}
can be either full path like \texttt{\$HOME/.ecru/someprof.conf} or just a
filename like \texttt{someprof.conf}, \Prog{ecru-config} cuts down the path and leaves
only filename. It's done that way so it were handy to use shell path completion.

\subsection{Hooks}
You can use \texttt{-h} to list hooks. Check out the options section to know more
details on this option.

\section{Options}

\begin{Description}\setlength{\itemsep}{0cm}
\item[\Opt{-l}] List configuration profiles.
\item[\OptArg{-s}{path} Switch current configuration profile to \Arg{path}.
\item[\Opt{-v}] Show ecru version.
\item[\OptArg{-h}{hook}] List hooks of type \Arg{hook}. Currently the following
values of \Arg{hook} are supported:
	\begin{itemize}
		\item \texttt{pre} - pre-hooks
		\item \texttt{post} - post-hooks
	\end{itemize}
\end{Description}

\section{Files}

\begin{Description}\setlength{\itemsep}{0cm}
\item[\File{\$HOME/.ecru}] Configuration directory root.
\item[\File{\$HOME/.ecru/default.conf}] Default configuration profile.
\item[\File{\$HOME/.ecru/current}] Pointer to the default configuration
profile, points to \texttt{default.conf} by default.
\item[\File{\$HOME/.ecru/hooks}] Hooks directory.
\item[\File{\$HOME/.ecru/hooks/pre}] Pre-hooks directory.
\item[\File{\$HOME/.ecru/hooks/post}] Post-hooks directory.
\item[\File{\$HOME/.ecru/templates}] Templates directory.
\end{Description}

\section{See Also}

\Cmd{ecru-post}{1}, \Cmd{ecru-edit}{1}, \Cmd{ecru-list}{1}, \Cmd{ecru-info}{1}
