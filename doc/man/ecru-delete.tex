\documentclass[english]{article}
\usepackage[latin1]{inputenc}
\usepackage{babel}
\usepackage{verbatim}
\usepackage[fancy]{ecru}

\setVersion{0.1.0}

\begin{document}

\begin{Name}{1}{ecru-delete}{Roman Bogorodskiy}{ecru}{ecru-delete}
	\Prog{ecru-delete} is a tool for deleting posts.
\end{Name}

\section{Synopsis}

\Prog{ecru-delete} \Arg{id1 id2 id3 ... idN}

\Prog{ecru-delete} \Opt{-v}


\section{Description}
\Prog{ecru-delete} is a tool for deleting entries from the journal. It takes
a list of item ids to delete and deletes them. You can obtain these ids by
calling \Cmd{ecru-list}{1}.

\section{Options}

\begin{Description}\setlength{\itemsep}{0cm}
\item[\Opt{-v}] Show ecru version.
\end{Description}

\section{See Also}

\Cmd{ecru-post}{1}, \Cmd{ecru-edit}{1}, \Cmd{ecru-list}{1}, \Cmd{ecru-info}{1}
